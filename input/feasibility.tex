\chapter{FeasibilityStudy}

\section{Feasibility}
Feasibility study aims to uncover the strengths and weaknesses of
a project. These are some feasibility factors by which we can used to determine that the project is feasible or not:
\begin{itemize}
	\item {\bf{Technical feasibility}}: Technological feasibility is carried out to determine whether the project has the capability, in terms of software, hardware, personnel to handle and fulfill the user requirements. This whole project is based on Open Source Environment and is part of an open source software which would be deployed on any OS.
	\item {\bf{Economic feasibility}}: OpenSCAD's multi-threaded compile and render is also Economically feasible with as It could be developed and maintain with zero cost as It is supported by Open source community. Plus This project is started with no intention of having any economic gain but still there is an option for donations. Economic Feasibility which can be categorized as follows:
		\begin{enumerate}
			\item Development costs.
			\item Operating costs.
		\end{enumerate}														
\end{itemize}


\section{Significance of Project}
Speed, concurrency, efficiency and better management of resources go a long way in improving any products' performance, usability and popularity. The same can certainly be said about an open source project like OpenSCAD.
Our project will explore above aspects for OpenSCAD.


\section{Objectives of the Project}
Objective of this project are following:
\begin{itemize}
\item Explore different OS independent ways of parallizing the evaluation.
\item Support for multi-threaded evaluation, possibly with some limitations to handle non-thread-safe library calls
\item Thread safe cache infrastructure
\item Support for safe halting of the render process.
\item Fixing some grammer related oversights in the modelling language.
\item Installing warning mechanism in case of there being multiple pseudo node tags in the model.
\end{itemize}
