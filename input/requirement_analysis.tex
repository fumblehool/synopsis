\chapter{Requirement Analysis}

\section{Feasibility Study}
Feasibility study aims to uncover the strengths and weaknesses of
a project. These are some feasibility factors by which we can used to determine that the project is feasible or not:
    \subsubsection{Technical Feasibility}
Technical feasibility is one of the first studies that must be conducted after the project has been identified. In large engineering projects consulting agencies that have large staffs of engineers and technicians conduct technical studies dealing with the projects. In individual agricultural projects financed by local agricultural credit corporations, the technical staff composed of specialized agricultural engineers, irrigation and construction engineers, and other technicians are responsible for conducting such feasibility studies.\\ The Technical feasibility assessment is focused on gaining an understanding of the present technical resources of the organization and their applicability to the expected needs of the proposed system. It is an evaluation of the hardware and software and how it meets the need of the proposed system. This assessment is based on an outline design of system requirements, to determine whether the company has the technical expertise to handle completion of the project. When writing a feasibility report, the following should be taken to consideration:
\begin{itemize}
    \item A brief description of the business to assess more possible factors which could affect the study
    \item The part of the business being examined
    \item The human and economic factor
    \item The possible solutions to the problem
\end{itemize}
 \subsubsection{Economic Feasibility}
The purpose of the economic feasibility assessment is to determine the positive economic benefits to the organization that the proposed system will provide. It includes quantification and identification of all the benefits expected. This assessment typically involves a cost/ benefits analysis.\\

Economic feasibility is the cost and logistical outlook for a business project or endeavor. Prior to embarking on a new venture, most businesses conduct an economic feasibility study, which is a study that analyzes data to determine whether the cost of the prospective new venture will ultimately be profitable to the company. Economic feasibility is sometimes determined within an organization, while other times companies hire an external company that specializes in conducting economic feasibility studies for them.\\

The purpose of business in a capitalist society is to turn a profit, or to earn positive income. While some ideas seem excellent when they are first presented, they are not always economically feasible. That is, that they are not always profitable or even possible within a company's budget. Since companies often determine their budget's several months in advance, it is necessary to know how much of the budget needs to be set aside for future projects. Economic feasibility helps companies determine what that dollar amount is before a project is ultimately approved. This allows companies to carefully manage their money to insure the most profitable projects are undertaken. Economic feasibility also helps companies determine whether or not revisions to a project that at first seems unfeasible will make it feasible.\\

The developing system must be justified by cost and benefit. Criteria to ensure that effort is concentrated on project, which will give best, return at the earliest. One of the factors, which affect the development of a new system, is the cost it would require. Economic feasibility determines whether the required software is capable of generating financial gains for an organization. It involves the cost incurred on the software development team, estimated cost of hardware and software, cost of performing feasibility study, and so on. For this, it is essential to consider expenses made on purchases (such as hardware purchase) and activities required to carry out software development. In addition, it is necessary to consider the benefits that can be achieved by developing the software. Software is said to be economically feasible if it focuses on the issues listed below.
\begin{itemize}
    \item Cost incurred on software development to produce long-term gains for an organization.
    \item Cost required to conduct full software investigation (such as requirements elicitation and requirements analysis).
    \item Cost of hardware, software, development team, and training.
\end{itemize}

The following are some of the important financial questions asked during preliminary investigation:
\begin{itemize}
    \item The costs conduct a full system investigation.
    \item The cost of the hardware and software.
    \item The benefits in the form of reduced costs or fewer costly errors.
\end{itemize}


\section{Software Requirements}
\begin{itemize}
    \item \textbf{Python 2.7} - Python is a programming language. It’s used for many different applications. It’s used in some high schools and colleges as an introductory programming language because Python is easy to learn, but it’s also used by professional software developers at places such as Google, NASA, and Lucasfilm Ltd. It is used for Research Work in the field of Machine Learning. 
    \item \textbf{Python pip} - pip is a package management system used to install and manage software packages written in Python. Many packages can be found in the default source for packages and their dependencies — Python Package Index. We will use it to install required python packages.
    \item \textbf{Python packages} - 
        \begin{itemize}
            \item \textbf{Pandas} - The Pandas module is a high performance, highly efficient, and high level data analysis library. We will use it to analyse our Dataset.
            \item \textbf{Sklearn} - Open Sourced simple and efficient tool for data mining and data analysis.
            \item \textbf{Keras} - Keras is a high-level neural networks API, written in Python and capable of running on top of TensorFlow, CNTK, or Theano.
            \item \textbf{Lxml} -  The most feature-rich and easy-to-use library for processing XML and HTML in the Python language.
        \end{itemize}
\end{itemize}

\section{Predictive Analytic Requirements}
Building a decision requirements model 
to specify business understanding at the very beginning 
of 
a predictive analytic project allows the creation of predictive analytic requirements that:
\begin{enumerate} 
    \item Describe a clear target for the project. The decisions that the predictive analytic will 
influence are specified. The decision requirements diagram links these decisions to the ultimate business metrics or objectives that will be impacted by the analytic.
    
    \item Identifythe analytics to be developed. Each piece of analytic knowledge can be described along with the information to be analyzed to produce it.
    
    \item Is specific about which decisions are being influenced.
The decision requirements 
diagram shows w
hich part of the 
decision making is influenced, exactly, 
by each 
predictive analytic model being developed and what other factors influence that 
decision
-
making.
    
    \item Is specific about deployment.
The links for the decisions involved show w
hich 
organizations will be involved
, w
hich business processes will be impacted
and w
hich 
systems will have to be altered.

A complete set of requirements for a predictive analytic project 
should include other project 
details such as executive sponsor, timeline, resources, planned analytic approach
, 
etc. The 
d
ecision 
r
equirements 
model allows the business problem being addressed by the project to be described more precisely, but it does not replace these other elements.
    
\end{enumerate}

\section{Software Requirements:}
\begin{enumerate}
\item Python 2.7
\item Python pip
\item Anaconda
\end{enumerate}

\section{Python Packages Dependencies}
\begin{enumerate}
\item Pandas
\item Sklearn
\end{enumerate}
